\href{https://travis-ci.org/Erriez/ErriezSerialTerminal}{\tt }

This is a universal Serial Terminal library for Arduino to parse A\+S\+C\+II commands and arguments.



\subsection*{Hardware}

Any Arduino hardware with a serial port.

\subsection*{Examples}


\begin{DoxyItemize}
\item Examples $\vert$ Erriez Serial Terminal $\vert$ \href{https://github.com/Erriez/ErriezSerialTerminal/blob/master/examples/GettingStarted/GettingStarted.ino}{\tt Getting\+Started}
\end{DoxyItemize}

\subsection*{Documentation}


\begin{DoxyItemize}
\item \href{https://erriez.github.io/ErriezSerialTerminal}{\tt Online H\+T\+ML}
\item \href{https://github.com/Erriez/ErriezSerialTerminal/raw/gh-pages/latex/ErriezSerialTerminal.pdf}{\tt Download P\+DF}
\end{DoxyItemize}

\subsection*{Usage}

{\bfseries Initialization}

Create a Serial Terminal object. This can be initialized with optional newline and delimiter characters.

Default newline character\+: {\ttfamily \textquotesingle{}\textbackslash{}n\textquotesingle{}} Default delimiter character\+: {\ttfamily Space}


\begin{DoxyCode}
1 \{c++\}
2 #include <SerialTerminal.h>
3 
4 // Newline character '\(\backslash\)r' or '\(\backslash\)n'
5 char newlineChar = '\(\backslash\)n'; 
6 // Separator character between commands and arguments
7 char delimiterChar = ' ';
8 
9 // Create serial terminal object
10 SerialTerminal term(newlineChar, delimiterChar);
11 
12 
13 void setup()
14 \{
15     // Initialize serial port
16     Serial.begin(115200);
17 
18     // Initialize the built-in LED
19     pinMode(LED\_BUILTIN, OUTPUT);
20     digitalWrite(LED\_BUILTIN, LOW);
21 \}
\end{DoxyCode}
 {\bfseries Register new commands}

Commands must be registered at startup with a corresponding {\ttfamily callback handler} . This registers the command only, excluding arguments.

The callback handler will be called when the command has been received including the newline character.

An example of registering multiple commands\+:


\begin{DoxyCode}
1 \{c++\}
2 void setup()
3 \{
4     ...
5 
6     // Add command callback handlers
7     term.addCommand("?", cmdHelp);
8     term.addCommand("help", cmdHelp);
9     term.addCommand("on", cmdLedOn);
10     term.addCommand("off", cmdLedOff);
11 \}
12 
13 void cmdHelp()
14 \{
15     // Print usage
16     Serial.println(F("Serial terminal usage:"));
17     Serial.println(F("  help or ?          Print this usage"));
18     Serial.println(F("  on                 Turn LED on"));
19     Serial.println(F("  off                Turn LED off"));
20 \}
21 
22 void cmdLedOn()
23 \{
24     // Turn LED on
25     Serial.println(F("LED on"));
26     digitalWrite(LED\_BUILTIN, HIGH);
27 \}
28 
29 void cmdLedOff()
30 \{
31     // Turn LED off
32     Serial.println(F("LED off"));
33     digitalWrite(LED\_BUILTIN, LOW);
34 \}
\end{DoxyCode}


{\bfseries Set default handler}

Optional\+: The default handler will be called when the command is not recognized.


\begin{DoxyCode}
1 \{c++\}
2 void setup()
3 \{   
4     ...
5 
6     // Set default handler for unknown commands
7     term.setDefaultHandler(unknownCommand);
8 \}
9 
10 void unknownCommand(const char *command)
11 \{
12     // Print unknown command
13     Serial.print(F("Unknown command: "));
14     Serial.println(command);
15 \}
\end{DoxyCode}


{\bfseries Read from serial port}

Read from the serial port in the main loop\+:


\begin{DoxyCode}
1 \{c++\}
2 void loop()
3 \{
4     // Read from serial port and handle command callbacks
5     term.readSerial();
6 \}
\end{DoxyCode}


{\bfseries Get next argument}

Get pointer to next argument in serial receive buffer\+:


\begin{DoxyCode}
1 \{c++\}
2 char *arg;
3 
4 // Get next argument
5 arg = term.getNext();
6 if (arg != NULL) \{
7     Serial.print(F("Argument: "));
8     Serial.println(arg);
9 \} else \{
10     Serial.println(F("No argument"));
11 \}
\end{DoxyCode}


{\bfseries Get remaining characters}

Get pointer to remaining characters in serial receive buffer\+:


\begin{DoxyCode}
1 \{c++\}
2 char *arg;
3 
4 // Get remaining characters
5 arg = term.getRemaining();
6 if (arg != NULL) \{
7     Serial.print(F("Remaining: "));
8     Serial.println(arg);
9 \}
\end{DoxyCode}


{\bfseries Clear buffer}

Optional\+: The serial receive buffer can be cleared with the following call\+:


\begin{DoxyCode}
1 \{c++\}
2 term.clearBuffer();
\end{DoxyCode}


\subsection*{Library configuration}

{\ttfamily \hyperlink{_serial_terminal_8h}{Serial\+Terminal.\+h}} contains the following configuration macro\textquotesingle{}s\+:


\begin{DoxyItemize}
\item {\ttfamily S\+T\+\_\+\+R\+X\+\_\+\+B\+U\+F\+F\+E\+R\+\_\+\+S\+I\+ZE} \+: The default serial receive buffer size is 32 Bytes. This includes the command and arguments, excluding the {\ttfamily \textquotesingle{}\textbackslash{}0\textquotesingle{}} character.
\item {\ttfamily S\+T\+\_\+\+N\+U\+M\+\_\+\+C\+O\+M\+M\+A\+N\+D\+\_\+\+C\+H\+A\+RS}\+: The default number of command characters is 8 Bytes, excluding the {\ttfamily \textquotesingle{}\textbackslash{}0\textquotesingle{}} character.
\end{DoxyItemize}

\subsection*{Library dependencies}


\begin{DoxyItemize}
\item None.
\end{DoxyItemize}

\subsection*{Library installation}

Please refer to the \href{https://github.com/Erriez/ErriezArduinoLibrariesAndSketches/wiki}{\tt Wiki} page.

\subsection*{Other Arduino Libraries and Sketches from Erriez}


\begin{DoxyItemize}
\item \href{https://github.com/Erriez/ErriezArduinoLibrariesAndSketches}{\tt Erriez Libraries and Sketches} 
\end{DoxyItemize}